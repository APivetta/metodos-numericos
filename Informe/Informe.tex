\documentclass[a4paper,10pt]{article}
\usepackage[spanish]{babel} %Paquetes de idioma
\usepackage[latin1]{inputenc} %Paquetes de idioma
\usepackage{graphicx} % Paquete para ingresar gráficos
\usepackage{hyperref}
\usepackage{fancybox}
\usepackage{color}
\usepackage{listings}
\lstset{language=Octave,xleftmargin=.20in}

%Encabezado y Pié de página
\usepackage{fancyhdr} % Paquete para encabezados y pie de página
\pagestyle{fancy} % Sin esta línea no se imprimiría el encabezado en todas las páginas

\fancyhf{} %  Borra el encabezado anterior (Por defecto escribe el títutlo de la sección en la que se encuentra la hoja
\setlength{\headheight}{22.55pt}
\fancyhead[L]{
	{\textsf{Universidad CAECE \\ Lenguajes de Programaci\'on - 2$^{\circ}$C 2016}}
}

\fancyhead[R]{\thepage}

\renewcommand{\footrulewidth}{0.4pt} % Ajusta el tamaño de las líneas separadoras en el pié de página
\renewcommand{\headrulewidth}{0.4pt} % Ajusta el tamaño de las líneas separadoras en el encabezado

\fancyfoot[L]{
%	{\textsf{TP N$^{\circ}$1: ...}} \\
	{\textsf{992850 - Agust\'in Pivetta}}
	}
		

%Carátula del Trabajo
\title{ \author{} % Lo pongo para que el warning no moleste :p
\setlength{\unitlength}{1cm} %  Especifica la unidad de trabajo
\thispagestyle{empty}
\begin{picture}(18,0)
\put(0,0){\includegraphics[width=5.34cm, height=1.5cm]{logo.png}}
%\put(10.5,0){\includegraphics[width=3cm, height=3cm]{Logo2.png}}
\end{picture}
\\[3.5cm]
\begin{center}
	\textbf{{\Huge Lenguajes de Programaci\'on}}\\[2cm]

	{2$^{\circ}$ Cuatrimestre 2016}\\[0.5cm]
	{992850 - Agust\'in Pivetta}\\[2.5cm]
\end{center}

\date{} % Hace que no se imprima la fecha en la cual se compilo el .tex
 }

\begin{document}
	\maketitle %Hace que el título anterior sea el principal del documento
	\newpage

	\tableofcontents %Esta línea genera un indice a partir de las secciones y subsecciones creadas en el documento
	\newpage

	\section{Introducci\'on}
		El objetivo del presente informe consiste en mostrar los algoritmos desarrollados a lo largo del cuatrimestre con sus correspondientes resultados de ejecuci\'on. Todos los algoritmos, junto a este informe, se encuentran en \url{https://github.com/APivetta/metodos-numericos} 

	\section{Aproximaci\'on num\'erica y errores}
		Se prueban dos m\'etodos para resolver la expresi\'on: $I_n = \int_{0}^{1} \frac{x^n}{x + 10} dx$

		\subsection{M\'etodo con mayor error}
 			\lstinputlisting{../Error/in_malo.m}
 			Resultado:
 			\lstinputlisting{../Error/in_malo.out}

		\subsection{M\'etodo con menor error}
 			\lstinputlisting{../Error/in_bueno.m}
 			Resultado:
 			\lstinputlisting{../Error/in_bueno.out}

	\section{Ceros de funciones}

		\subsection{Punto Fijo}
 			\lstinputlisting{../Ceros/puntoFijo.m}
 			Resultado:
 			\lstinputlisting{../Ceros/puntoFijo.out}

		\subsection{Regula Falsi}
 			\lstinputlisting{../Ceros/regulaFalsi.m}
 			Resultado:
 			\lstinputlisting{../Ceros/regulaFalsi.out}

		\subsection{Secante}
 			\lstinputlisting{../Ceros/secante.m}
 			Resultado:
 			\lstinputlisting{../Ceros/secante.out}

		\subsection{Newton Raphson}
 			\lstinputlisting{../Ceros/newtonRaphson.m}
 			Resultado:
 			\lstinputlisting{../Ceros/newtonRaphson.out}

	\section{Interpolaci\'on}

		\subsection{Lagrange}
 			\lstinputlisting{../Interpolacion/lagrange.m}
 			Resultado:
 			\lstinputlisting{../Interpolacion/lagrange.out}

	\section{Aproximaci\'on de funciones}

		\subsection{Regresion lineal}
 			\lstinputlisting{../Aproximacion/regresionLineal.m}
 			Resultado:
 			\lstinputlisting{../Aproximacion/regresionLineal.out}

	\section{Integraci\'on num\'erica}

		\subsection{Trapecios}
 			\lstinputlisting{../Integracion/trapecio.m}
 			Resultado:
 			\lstinputlisting{../Integracion/trapecio.out}

		\subsection{Romberg}
 			\lstinputlisting{../Integracion/romberg.m}
 			Resultado:
 			\lstinputlisting{../Integracion/romberg.out}

		\subsection{Simpson}
 			\lstinputlisting{../Integracion/simpson.m}
 			Resultado:
 			\lstinputlisting{../Integracion/simpson.out}

	\section{Ecuaciones diferenciales}

		\subsection{Euler}
 			\lstinputlisting{../EDO/euler.m}
 			Resultado:
 			\lstinputlisting{../EDO/euler.out}

		\subsection{Euler Mejorado}
 			\lstinputlisting{../EDO/eulerMejorado.m}
 			Resultado:
 			\lstinputlisting{../EDO/eulerMejorado.out}

		\subsection{Runge-Kutta}
 			\lstinputlisting{../EDO/rungeKutta.m}
 			Resultado:
 			\lstinputlisting{../EDO/rungeKutta.out}

	\section{Sistemas de ecuaciones}

		\subsection{Gauss}
 			\lstinputlisting{../SistemasDeEcuaciones/gauss.m}
 			Resultado:
 			\lstinputlisting{../SistemasDeEcuaciones/gauss.out}

\end{document}


